\begin{figure}[!htb]
\begin{tikzpicture}[remember picture]
  \node (G) {
  	\begin{tikzpicture}[main node/.style={circle,draw,minimum size=.8cm,inner sep=0pt}]
		\node[main node] (G1) {$a$};
		\node[main node] (G2) [right = 1cm of G1] {$b$};
		\node[main node] (G3) [right = 1cm of G2] {$c$};
		\node[main node] (G4) [right = 1cm of G3] {$d$};
		\node[main node] (G5) [right = 1cm of G4] {$e$};
		\node[main node] (G6) [below = 1.5cm of G2] {$f$};
		\node[main node] (G7) [below = 1.5cm of G3] {$g$};
		
		\path[draw,thick]
		(G1) edge node {} (G2)
		(G2) edge node {} (G3)
		(G3) edge node {} (G4)
		(G4) edge node {} (G5)
		(G2) edge node {} (G6)
		(G3) edge node {} (G7)
		(G6) edge node {} (G7)
		(G4) edge [bend right=50] node {} (G2);
  	\end{tikzpicture}
  };
	\node at (0,-2.5) {Graph G};
    %%
    \node(H)[right=of G]{
	\begin{tikzpicture}[main node/.style={circle,draw,minimum size=.8cm,inner sep=0pt}]
    \node[main node] (H1) {$1$};
    \node[main node] (H2) [right = 1cm  of H1]  {$2$};
    \node[main node] (H3) [right = 1cm  of H2] {$3$};
    \node[main node] (H4) [right = 1cm  of H3] {$4$};

    \path[draw,thick]
     (H1) edge node {} (H2)
     (H2) edge node {} (H3)
     (H3) edge node {} (H4)
     (H1) edge [bend left=50] node {} (H3)
    ;
    \end{tikzpicture}   
    };   
    \node at (8, -2.5) {Graph H};
    \path[draw,dashed]
    	(H1) edge [bend right=80, ->] node {} (G2)
    	(H2) edge [bend right=80, ->] node {} (G3)
    	(H3) edge [bend right=80, ->] node {} (G4)
    	(H4) edge [bend right=80, ->] node {} (G5)
    ;
\end{tikzpicture}
\caption{Example of a subgraph isomorphism}
\label{fig:subgraphisoexample}
\end{figure}